% !TeX root = presentation.tex
%% presentation.tex
%% tout ce qui concerne la présentation
\usepackage{hobsub-generic}

\usepackage[T1]{fontenc}
% \usepackage{CormorantGaramond}
\usepackage{gillius2}
\usepackage[utf8]{inputenc}
\usepackage[french]{babel}

\usepackage[strict]{changepage}

\usepackage{marginnote}
\usepackage{makeidx}
\usepackage{graphicx}

\usepackage{titlesec}
\usepackage{multicol}
\usepackage{FiraMono}

\usepackage{xspace}

\xspaceaddexceptions{\,}
\xspaceaddexceptions{\.}

\usepackage{listings}
\usepackage{color}
\usepackage{xparse} % pour les nouveaux environnements avec paramètres
\usepackage[parfill]{parskip} % Activate to begin paragraphs with an empty line rather than an indent

%%% PACKAGES
\usepackage{booktabs} % for much better looking tables
\usepackage{array} % for better arrays (eg matrices) in maths
\usepackage{paralist} % very flexible & customisable lists (eg. enumerate/itemize, etc.)
\usepackage{verbatim} % adds environment for commenting out blocks of text & for better verbatim
\usepackage{subfig} % make it possible to include more than one captioned figure/table in a single float

% \addbibresource{bibliographie.bib}
% \newcommand{\docite}[1] {\footnotesize \textbf{\emph{ (\cite{#1}) }}\normalsize}
%% This turns references into clickable hyperlinks.
\usepackage[bookmarks,backref=page,linkcolor=black]{hyperref} %,colorlinks
\hypersetup{%
  pdfauthor = {},
  pdftitle = {},
  pdfsubject = {},
  pdfkeywords = {},
  colorlinks=true,
  linkcolor= black,
  citecolor= black,
  pageanchor=true,
  urlcolor = black,
  plainpages = false,
  linktocpage
}

%% format des paragraphes
\setlength{\parindent}{0cm}
\setlength{\parskip}{4mm}
\linespread{1.2}

\usepackage{caption}
\usepackage{boxhandler}

\definecolor{mygreen}{rgb}{0,0.6,0}
\definecolor{mygray}{rgb}{0.5,0.5,0.5}
\definecolor{mymauve}{rgb}{0.58,0,0.82}
%% programming styles
\definecolor{trust}{rgb}{0.14,0.51,0.14}
\definecolor{lgray}{rgb}{0.96,0.96,0.96}
\definecolor{hgray}{rgb}{0.31,0.31,0.31}
\definecolor{gray}{rgb}{0.41,0.41,0.41}

\definecolor{kword}{rgb}{0.01,0.01,0.71}
\definecolor{morekword}{rgb}{0.1,0.7,0.01}
\definecolor{colVarName}{rgb}{0.01,0.71,0.01}
\definecolor{colVarValue}{rgb}{0.71,0.01,0.01}


\lstset{%
  backgroundcolor=\color{white},   % choose the background color; you must add \usepackage{color} or \usepackage{xcolor}; should come as last argument
  basicstyle=\footnotesize\ttfamily,        % the size of the fonts that are used for the code
  breakatwhitespace=false,         % sets if automatic breaks should only happen at whitespace
  breaklines=true,                 % sets automatic line breaking
  captionpos=b,                    % sets the caption-position to bottom
  commentstyle=\color{mygreen},    % comment style
  %deletekeywords={...},            % if you want to delete keywords from the given language
  %% escapeinside={\%*}{*)},          % if you want to add LaTeX within your code
  extendedchars=true,              % lets you use non-ASCII characters; for 8-bits encodings only, does not work with UTF-8
  frame=single,	                   % adds a frame around the code
  keepspaces=true,                 % keeps spaces in text, useful for keeping indentation of code (possibly needs columns=flexible)
  keywordstyle=\color{blue},       % keyword style
  language=bash,                 % the language of the code
  % morekeywords={*,...},           % if you want to add more keywords to the set
  numbers=none,                    % where to put the line-numbers; possible values are (none, left, right)
  numbersep=5pt,                   % how far the line-numbers are from the code
  numberstyle=\tiny\color{mygray}, % the style that is used for the line-numbers
  rulecolor=\color{black},         % if not set, the frame-color may be changed on line-breaks within not-black text (e.g. comments (green here))
  showspaces=false,                % show spaces everywhere adding particular underscores; it overrides 'showstringspaces'
  showstringspaces=false,          % underline spaces within strings only
  showtabs=false,                  % show tabs within strings adding particular underscores
  stepnumber=2,                    % the step between two line-numbers. If it's 1, each line will be numbered
  stringstyle=\color{mymauve}\textit,     % string literal style
  tabsize=2,	                   % sets default tabsize to 2 spaces
  title=\lstname}                   % show the filename of files included with \lstinputlisting; also try caption instead of title

\lstset{%
    % language={[LaTeX]TeX},
    % alsolanguage={PGF/TikZ},
    frame=single,
    framesep=\fboxsep,
    framerule=\fboxrule,
    rulecolor=\color{red},
    xleftmargin=\dimexpr\fboxsep+\fboxrule,
    xrightmargin=\dimexpr\fboxsep+\fboxrule,
    breaklines=true,
    basicstyle=\small\tt,
    keywordstyle=\color{blue}\sf\textbf,
    identifierstyle=\color{teal}\textit,
    commentstyle=\color{darkgray}\textbf,
    backgroundcolor=\color{yellow!10},
    tabsize=2,
    columns=flexible,
}

\lstdefinestyle{shell}{tabsize=4,
	basicstyle=\fontsize{10}{12}\selectfont\ttfamily,
	language=bash,
	numbers=left,numberstyle=\tiny\color{gray},
	classoffset=0,
	keywordstyle=\bfseries\color{kword},
	classoffset=1,
	morekeywords={cat,cut,sort,uniq,head,tail,tr,IFS,sudo,find,sh,bash,stat,xargs,date,grep,egrep,sed,basename,dirname,which},
	keywordstyle=\bfseries\color{morekword},
	classoffset=0,
	backgroundcolor=\color{lgray},
	breaklines=true,
  	showspaces=false,
  	showstringspaces=false
}


\DeclareCaptionFont{white}{\color{white}}
\DeclareCaptionFormat{listing}{%
  \colorbox[cmyk]{0.43, 0.35, 0.35,0.01 }{%
    \parbox{0.8\textwidth}{\hspace{15pt}#1#2#3}
  }
}
\captionsetup[lstlisting]{format=listing, labelfont=white, textfont=white, singlelinecheck=false, margin=2em, font={footnotesize}}


%% part redefinition
\newcommand\headerdisplay[1]{%
   \huge
   \vskip.5\baselineskip
   \filcenter\MakeUppercase{#1}%
   \vskip.0\baselineskip
}
%% \NewCoffin\mytmpa
%% \NewCoffin\mytmpb
%% \newcommand\placeabove[3][0pt]{%
  %% \SetHorizontalCoffin\mytmpa{#2}%
  %% \SetHorizontalCoffin\mytmpb{#3}%
  %% \JoinCoffins*\mytmpb[hc,t]\mytmpa[hc,b](0pt,#1)%
  %% \TypesetCoffin\mytmpb
%% }

\renewcommand\thepart{\arabic{part}}
\titleclass{\part}{top} % make part like a chapter
\titleformat{\part}[frame]
   {\normalfont}
   %% {\filcenter\placeabove[2\baselineskip]{\Large }{\huge\enspace\thepart\enspace}}
   {}
   {0pt}
   {\headerdisplay}
\titlespacing*{\part}{0pt}{0pt}{20pt}

%%% HEADERS & FOOTERS
\newif\ifwithfancyhdr	\withfancyhdrtrue
\ifwithfancyhdr
	\usepackage{fancyhdr} % This should be set AFTER setting up the page geometry

	%% \pagestyle{headings} % options: empty , plain , fancy
	%% \renewcommand{\headrulewidth}{0pt} % customise the layout...
	\lhead{Optimiser son Unix (Linux, BSD\ldots)}\chead{}\rhead{\currentname}
	\lfoot{}\cfoot{\thepage}\rfoot{}
\else
	\pagestyle{plain} % options: empty , plain , fancy
\fi

%% formattage de la table des matières : indentation en fonction du niveau
\setcounter{tocdepth}{3}% to get subsubsections in toc
%% pour l'apparence des listes
\frenchbsetup{StandardItemLabels=true, CompactItemize=false, ReduceListSpacing=false}

\usepackage{wrapfig}

\newcommand\nicepicture[3] {
	\begin{wrapfigure}{R}{0.45\textwidth}
		\centering
		\includegraphics[width=0.9\linewidth]{#1}
		\caption{\emph{#2} \\
		\footnotesize Source: #3}
	\end{wrapfigure}
}

\newcommand\smallpicture[3] {
	\begin{wrapfigure}{L}{0.45\textwidth}
		\centering
		\includegraphics[width=0.9\linewidth]{#1}
		\caption{\emph{#2} \\
		\footnotesize Source: #3}
	\end{wrapfigure}
}

\newcommand\leftpicture[3] {
	\begin{wrapfigure}{L}{0.45\textwidth}
		\centering
		\includegraphics[width=0.9\linewidth]{#1}
		\caption{\emph{#2} \\
		\footnotesize Source: #3}
	\end{wrapfigure}
}

%% #1 : fichier image
%% #2 : caption
%% #3 : la source
\newcommand\rightpicture[3] {
	\begin{wrapfigure}{R}{0.45\textwidth}
		\includegraphics[width=0.9\linewidth]{#1}
		\caption{\emph{#2} \\
		\footnotesize Source: #3}
	\end{wrapfigure}
}




\usepackage{bold-extra}

\NewDocumentEnvironment{Quotebis}{m}{%
	\begin{adjustwidth}{1.5cm}{1.5cm}
		\begin{itshape}
}
{%
		\end{itshape}
		\begin{center}
			\begin{minipage}{72mm}
				\center
				 \footnotesize \cf #1
			\end{minipage}
		\end{center}
	\end{adjustwidth}\medskip
}


\newenvironment{Quote} {%
		\begin{adjustwidth}{1.5cm}{1.5cm}
		\itshape
	}
	{%
		\end{adjustwidth}\medskip
	}



%% Macros
\newcommand{\inmargin}[1]{\marginnote{\scriptsize\textit{#1}\normalsize}}
\newcommand{\commande}[1]{\footnotesize\texttt{\textbf{#1}}\index{#1}\normalsize}
\newcommand{\code}[1]{\footnotesize\texttt{\emph{#1}}\index{#1}\normalsize}
