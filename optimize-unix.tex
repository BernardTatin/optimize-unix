% !TEX TS-program = pdflatex
% !TEX encoding = UTF-8 Unicode

% This is a simple template for a LaTeX document using the "article" class.
% See "book", "report", "letter" for other types of document.

\documentclass[12pt]{amsproc} % use larger type; default would be 10pt

\usepackage[french]{babel}
\usepackage[utf8]{inputenc} % set input encoding (not needed with XeLaTeX)

%% \usepackage{kpfonts}
%% \usepackage[default]{gillius}
%% \usepackage{Alegreya} %% Option 'black' gives heavier bold face 
%% \usepackage{lmodern}
%%% Examples of Article customizations
% These packages are optional, depending whether you want the features they provide.
% See the LaTeX Companion or other references for full information.

%%% PAGE DIMENSIONS
\usepackage{geometry} % to change the page dimensions
\geometry{a4paper} % or letterpaper (US) or a5paper or....
% \geometry{margin=2in} % for example, change the margins to 2 inches all round
% \geometry{landscape} % set up the page for landscape
%   read geometry.pdf for detailed page layout information

\usepackage{graphicx} % support the \includegraphics command and options
\usepackage{xparse} % pour les nouveaux environnements avec paramètres
\usepackage[parfill]{parskip} % Activate to begin paragraphs with an empty line rather than an indent

%%% PACKAGES
\usepackage{booktabs} % for much better looking tables
\usepackage{array} % for better arrays (eg matrices) in maths
\usepackage{paralist} % very flexible & customisable lists (eg. enumerate/itemize, etc.)
\usepackage{verbatim} % adds environment for commenting out blocks of text & for better verbatim
\usepackage{subfig} % make it possible to include more than one captioned figure/table in a single float
% These packages are all incorporated in the memoir class to one degree or another...

\usepackage{xspace}

\xspaceaddexceptions{\,}
\xspaceaddexceptions{\.}

\usepackage{listings}
\usepackage{courier}
\usepackage{color}
\usepackage{hyperref}

\hypersetup{unicode=false,          % non-Latin characters in Acrobat�s bookmarks
    pdftoolbar=true,        % show Acrobat�s toolbar?
    pdfmenubar=true,        % show Acrobat�s menu?
    pdffitwindow=false,     % window fit to page when opened
    pdfstartview={FitB},    % fits the bounding box of the page to the window
    pdftitle={OPTIMISER SON UTILISATION D’UNIX},    % title
    pdfsubject={OPTIMISER SON UTILISATION D’UNIX},   % subject of the document
    pdfauthor={Bernard Tatin},     % author
    pdfnewwindow=true,      % links in new window
    colorlinks=true,       % false: boxed links; true: colored links
    linkcolor=red,          % color of internal links (change box color with linkbordercolor)
    citecolor=green,        % color of links to bibliography
    filecolor=magenta,      % color of file links
    urlcolor=blue           % color of external links
}


%%% HEADERS & FOOTERS
\usepackage{fancyhdr} % This should be set AFTER setting up the page geometry

\pagestyle{plain} % options: empty , plain , fancy
%% \pagestyle{headings} % options: empty , plain , fancy
%% \renewcommand{\headrulewidth}{0pt} % customise the layout...
\lhead{Optimiser son Unix}\chead{}\rhead{\currentname}
\lfoot{}\cfoot{\thepage}\rfoot{}

%%% SECTION TITLE APPEARANCE
%% \usepackage{sectsty}
%% \allsectionsfont{\sffamily\mdseries\upshape} % (See the fntguide.pdf for font help)
% (This matches ConTeXt defaults)

%% formattage de la table des matières : indentation en fonction du niveau
\setcounter{tocdepth}{3}% to get subsubsections in toc

%\let\oldtocsection=\tocsection
%\let\oldtocsubsection=\tocsubsection
%\let\oldtocsubsubsection=\tocsubsubsection
%
%\renewcommand{\tocsection}[2]{\hspace{0em}\oldtocsection{#1}{#2}}
%\renewcommand{\tocsubsection}[2]{\hspace{1em}\oldtocsubsection{#1}{#2}}
%\renewcommand{\tocsubsubsection}[2]{\hspace{2em}\oldtocsubsubsection{#1}{#2}}


%% pour l'apparence des listes
\frenchbsetup{StandardItemLabels=true, CompactItemize=false, ReduceListSpacing=false}

%%% ToC (table of contents) APPEARANCE
%% \usepackage[nottoc,notlof,notlot]{tocbibind} % Put the bibliography in the ToC
%% \usepackage[titles,subfigure]{tocloft} % Alter the style of the Table of Contents
%% \renewcommand{\cftsecfont}{\rmfamily\mdseries\upshape}
%% \renewcommand{\cftsecpagefont}{\rmfamily\mdseries\upshape} % No bold!

\usepackage{wrapfig}

\newcommand\nicepicture[3] {
\begin{wrapfigure}{R}{0.45\textwidth}
\centering
\includegraphics[width=0.9\linewidth]{#1}
\caption{\emph{#2} \\
\footnotesize Source: #3}
\end{wrapfigure}
}

\newcommand\smallpicture[3] {
\begin{wrapfigure}{L}{0.45\textwidth}
\centering
\includegraphics[width=0.9\linewidth]{#1}
\caption{\emph{#2} \\
\footnotesize Source: #3}
\end{wrapfigure}
}

\newcommand\leftpicture[3] {
\begin{wrapfigure}{L}{0.45\textwidth}
\centering
\includegraphics[width=0.9\linewidth]{#1}
\caption{\emph{#2} \\
\footnotesize Source: #3}
\end{wrapfigure}
}

%% #1 : fichier image
%% #2 : caption
%% #3 : la source
\newcommand\rightpicture[3] {
\begin{wrapfigure}{R}{0.45\textwidth}
\includegraphics[width=0.9\linewidth]{#1}
\caption{\emph{#2} \\
\footnotesize Source: #3}
\end{wrapfigure}
}



\newcommand\Cf{\emph{Cf.}\xspace{}}
\newcommand\cf{\emph{cf.}\xspace{}}
\newcommand\ie{\emph{i.e.}\xspace{}}

\usepackage{bold-extra}

\newcommand\osname[1]{\textsc{\textbf{#1}}\index{#1@\textsc{#1}}\xspace}
\newcommand\netbsd{\osname{NetBSD}}
\newcommand\freebsd{\osname{FreeBSD}}
\newcommand\BSD{\osname{BSD}}

\newcommand\linux{\osname{Linux}}
\newcommand\cygwin{\osname{Cygwin}}
\newcommand\unix{\osname{Unix}}
\newcommand\multics{\osname{Multics}}
\newcommand\GNU{\osname{GNU}}
\newcommand\Debian{\osname{Debian}}
\newcommand\RedHat{\osname{RedHat}}

\newcommand\POSIX{\osname{POSIX}}

\newcommand\msdos{\osname{MS/DOS}}
\newcommand\windows{\osname{Windows}}

\newcommand\cmdname[1]{\emph{\texttt{#1}}\xspace\index{#1@\texttt{#1}}\xspace}
\definecolor{codecolor}{rgb}{0.08,0.4,0.07}
\newcommand\code[1]{\textcolor{codecolor}{\textbf{\cmdname{#1}}}\xspace}

\newcommand\sh{\cmdname{sh}}
\newcommand\bash{\cmdname{bash}}
\newcommand\zsh{\cmdname{zsh}}
\newcommand\csh{\cmdname{csh}}
\newcommand\tcsh{\cmdname{tcsh}}
\newcommand\ksh{\cmdname{ksh}}
\newcommand\awk{\cmdname{awk}}
\newcommand\shell{\emph{\textsc{shell}}\xspace{}}
\newcommand\shells{\emph{\textsc{shells}}\xspace{}}

\NewDocumentEnvironment{Quotebis}{m}{
	\begin{list}
		{}
		{\rightmargin0pt\leftmargin 2.5em}
	\item

	\item
		\begin{itshape}
}
{
		\end{itshape}
	\end{list}
	\begin{center}
		\begin{minipage}{72mm}
		\center
		 \footnotesize \cf #1
		\end{minipage}
		\end{center}
}


\newenvironment{Quote} {
	{\rightmargin0pt\leftmargin2.5em}
	\begin{itshape}
	}
	{
	\end{itshape}
}

%% programming styles
\definecolor{trust}{rgb}{0.14,0.51,0.14}
\definecolor{lgray}{rgb}{0.96,0.96,0.96}
\definecolor{hgray}{rgb}{0.31,0.31,0.31}
\definecolor{gray}{rgb}{0.41,0.41,0.41}

\definecolor{kword}{rgb}{0.01,0.01,0.71}
\definecolor{morekword}{rgb}{0.1,0.7,0.01}
\definecolor{colVarName}{rgb}{0.01,0.71,0.01}
\definecolor{colVarValue}{rgb}{0.71,0.01,0.01}
\definecolor{kword}{rgb}{0.01,0.01,0.71}


\lstdefinestyle{shell}{tabsize=4,
	basicstyle=\fontsize{10}{12}\selectfont\ttfamily,
	language=bash,
	numbers=left,numberstyle=\tiny\color{gray},
	classoffset=0,
	keywordstyle=\bfseries\color{kword},
	classoffset=1,
	morekeywords={cat,cut,sort,uniq,head,tail,tr,IFS,sudo,find,sh,bash,stat,xargs,date,grep,egrep,sed,basename,dirname,which},
	keywordstyle=\bfseries\color{morekword},
	classoffset=0,
	backgroundcolor=\color{lgray},
	breaklines=true,
  	showspaces=false,
  	showstringspaces=false
}

\renewcommand{\thepart}{ - \emph{\alph{part}}}

\usepackage{chngcntr}
\counterwithin{section}{part}
%%% END Article customizations
\makeindex
%%% The "real" document content comes below...

\title{Optimiser son utilisation d'Unix}
\author{Bernard TATIN \\
\tiny \\
\textit{bernard.tatin@outlook.fr}}
\makeindex


\date{2013/2017} % Activate to display a given date or no date (if empty),
         % otherwise the current date is printed

\linespread{1.1}
\begin{document}
 \begin{abstract}
 Ce document vient des tréfonds de l'espace temps. Il a débuté il y a bien plus de trois ans de cela, repris de manière plus systématique et se trouve fortement complété aujourd'hui.

 La première partie rappelle (rapidement) l'histoire et les concepts principaux des \shells{}.
 La deuxième partie est très orientée sur la recherche de \emph{qui a piraté ma machine} mais peut être
 d'une grande utilité pour les débutants. La troisième partie, quant à elle, se focalise sur les scripts.
 Une quatrième partie donnera des notions des outils indispensables pour utiliser correctement son système \unix{}.

 Ce document (\emph{en pleine réorganisation}) et ses sources en \LaTeX{} sont disponibles sur \href{https://github.com/BernardTatin/optimize-unix}{GitHub}.
 \end{abstract}
\maketitle
\tableofcontents

%\newpage
\part{l'histoire et les concepts}
% !TeX root = optimize-unix.tex

\section{Une histoire d'\unix}
Voici une (rapide) histoire d'\unix, choisie parmi d'autres, parmi celles qui évoluent avec le temps autant parce que des personnages hauts en couleur et ayant réussi à voler la vedette à de plus modestes collègues se font effacer eux-même par de plus brillants qu'eux, soit parce que, vieillissant ils se laissent aller à des confidences inattendues.

En nous basant sur \href{http://www.tuteurs.ens.fr/unix/histoire.html}{Brève histoire d'\unix}, on rappelle que \emph{AT\&T} travaillait à la fin des années 60, sur un système d'exploitation \href{http://fr.wikipedia.org/wiki/Multics}{\multics} qui devait révolutionner l'histoire de l'informatique. Si révolution il y eut, ce fut dans les esprits: de nombreux concepts de ce système ont influencés ses successeurs, dont \unix. Ken Thompson et Dennis Ritchie des fameux \emph{Bell Labs} et qui travaillaient (sans grande conviction, semble-t-il) sur \multics, décidèrent de lancer leur propre projet d'OS : 

\begin{Quote}
baptisé initialement UNICS (UNiplexed Information and Computing Service) jeu de mot avec "eunuchs' (eunuque) pour "un \multics emasculé", par clin d'œil au projet \multics, qu'ils jugeaient beaucoup trop compliqué. Le nom fut ensuite modifié en \unix\footnote{\cf l'article \href{http://fr.wikipedia.org/wiki/Multics}{\multics} de Wikipedia}.
\end{Quote}

\begin{Quote}
L'essor d'\unix est très fortement lié à un langage de programmation, le C. À l'origine, le premier \unix était écrit en assembleur, puis Ken Thompson crée un nouveau langage, le B. En 1971, Dennis Ritchie écrit à son tour un nouveau langage, fondé sur le B, le C. Dès 1973, presque tout \unix est réécrit en C. Ceci fait probablement d'\unix le premier système au monde écrit dans un langage portable, c'est-à-dire autre chose que de l'assembleur\footnote{\cf  \href{http://www.tuteurs.ens.fr/unix/histoire.html}{Brève histoire d'\unix}}.
\end{Quote}

Ce que j'ai surtout retenu de tout cela,  c'est qu'\unix a banalisé autant l'utilisation des stations de travail connectées en réseau que le concept de \emph{shell}, des système de fichiers hiérarchisés, des périphériques considérés comme de simples fichiers, concepts repris (et certainement améliorés) à \multics comme à d'autres. Pour moi, la plus grande invention d'\unix, c'est le langage C qui permet l'écriture des systèmes d'exploitations et des logiciels d'une manière très portable. N'oublions pas qu'aujourd'hui encore, C (mais pas C++) est un des langages les plus portable, même s'il commence à être concurrencé par Java par exemple.

% !TeX root = optimize-unix.tex

\section{Les \shells{}}
Un \shell{} est une \emph{coquille}, pour reprendre la traduction littérale, autour du système d'exploitation. Voici un magnifique diagramme (d'après ce que l'on trouve sur le WEB comme dans d'anciens ouvrages) donnant une idée du concept :

\leftpicture{diagrammes/ShellUnix.png}{shell \unix}{le WEB, ouvrages divers}

Entre mes débuts dans le monde de l'informatique et aujourd'hui, le concept de  \shell{} a quelque peu évolué. Certains qualifient l'explorateur de Windows comme un  \shell{}. Ont-ils raison? Certainement si l'on se réfère à l'image précédente : nos \emph{commandes} (clique, clique et reclique) envoyée au \emph{shell graphique} sont transmises au noyau qui nous renvoie, par l'intermédiaire du \emph{shell graphique}, de belles images. Il faut dire que l'explorateur Windows est le premier contact que l'utilisateur a avec sa machine.  Et sur l'article \href{http://fr.wikipedia.org/wiki/Interface_syst%C3%A8me}{interface système} de Wikipedia, on trouve cette définition:

\begin{Quote}
Une \emph{interface système }( \shell{} en anglais) est une couche logicielle qui fournit l'interface utilisateur d'un système d'exploitation. Il correspond à la couche la plus externe de ce dernier.
\end{Quote}

Ce même article cite les :
\begin{Quote}
\emph{shells graphiques} fournissant une interface graphique pour l'utilisateur (GUI, pour Graphical User Interface)
\end{Quote}

Dans le monde \unix{}, le concept de  \shell{} reste plus modeste, même si \emph{Midnight Commander} (\texttt{mc}) est parfois considéré comme un \shell{}:

\rightpicture{images/mc.png}{mc dans une session Cygwin}{mon PC}

Pour nous et dans tout ce qui suit, nous considérons comme  \shell{} :

\begin{Quotebis}{\href{http://fr.wikipedia.org/wiki/Shell_Unix}{ \shell{} \unix{}} sur Wikipedia}
un interpréteur de commandes destiné aux systèmes d'exploitation \unix{} et de type \unix{} qui permet d'accéder aux fonctionnalités internes du système d'exploitation. Il se présente sous la forme d'une interface en ligne de commande accessible depuis la console ou un terminal. L'utilisateur lance des commandes sous forme d'une entrée texte exécutée ensuite par le  \shell{}. Dans les différents systèmes \windows, le programme analogue est \code{command.com} ou \code{cmd.exe}.
\end{Quotebis}

\subsection{Le fonctionnement}
Le fonctionnement général est assez simple, surtout si l'on ne tient pas compte de la gestion des erreurs comme dans le graphique suivant qui peut être appliqué à tout bon interpréteur. Seuls les détails de \code{process command} et \code{execute command} vont réellement changer.

\rightpicture{diagrammes/shell-main-states.png}{ \shell{} : fonctionnement général}{créé avec \emph{yEd}}
L'exécution d'un programme suit l'algorithme :

\leftpicture{diagrammes/shell-exec.png}{ \shell{} : exécution d'un programme}{créé avec \emph{yEd}}
À noter que la commande \code{exec} se comporte différemment : elle correspond à l'appel système \code{exec}.

\subsection{Quelques \shells{} célèbres}
\subsubsection{\sh{}, le Bourne \shell{}}
L'ancêtre, toujours vivant et avec lequel sont écrits une grande majorité des scripts actuels. Son intérêt essentiel est justement l'écriture de scripts. Pour l'interaction, il est absolument \emph{nul} mais bien utile parfois pour dépanner.

\subsubsection{\csh{}, le C \shell{}}
Il se voulait le remplaçant glorieux de l'\emph{ancêtre} \sh{} avec une syntaxe considérée plus lisible car proche du C. Il est de plus en plus abandonné y compris par ses admirateurs les plus fervents, vieillissants dans la solitude la plus complète. Essayez d'écrire un script en csh d'un peu d'envergure sans faire de copié/collé! Il n'y a en effet pas de possibilité de créer des fonctions et, ce qui gêne peut-être encore plus les administrateurs système, il n'y a pas de gestion d'exception.  Cependant, il fût certainement le premier à proposer l'historique des commandes.

A noter qu'il fût crée par Bill Joy, l'un des fondateurs historiques de la société Sun Microsystems.

\subsubsection{\tcsh{} ou le \csh{} interactif}
Le pendant interactif du précédent. Il lui reste des afficionados qui aiment bien sa gestion de l'historique et de la ligne de commande.  Il est une \emph{extension} de \csh{}, \ie tout ce qui peut-être fait par \csh{} est fait par \tcsh{}. Sur de nombreux systèmes (Mac OS X  entre autre), ces deux \shell{}s pointent sur le même exécutable (avec un lien symbolique).

En séquence \emph{nostalgie}, je me souviens que c'est ce \shell{} interactif que j'utilisais sur mon premier \unix{}, en 87/88.

\subsubsection{\ksh{}, le Korn  \shell{}}
Initialement écrit pour \unix{} par David Korn au début des années 80, ce \shell{} a été repris par Microsoft pour Windows. Compatible avec \sh{}, il propose de nombreuses avancées comme beaucoup de fonctionnalités de \tcsh{},  des fonctions, des exceptions, des manipulations très évoluées de chaînes de caractères, ...

\subsubsection{\zsh{}, le Z  \shell{}}
C'est mon préféré pour l'interactivité, la complétion et bien d'autres choses encore dont il est capable depuis sa création ou presque. Comme \ksh{}, il est compilable en bytecode et propose des bibliothèques thématiques comme la couleur, les sockets, la gestion des dates...

\subsubsection{\bash{}, Bourne Again  \shell{}}
C'est le descendant le plus direct de \sh{}. C'est certainement le \shell{} le plus répandu dans le monde Linux aujourd'hui. 

Lors de ma découverte de Linux, je l'ai vite abandonné car il était très en retard pour la complétion en ligne de commande par rapport à d'autres, y compris \tcsh{} qui commençait pourtant à vieillir un peu. Il a fallu beaucoup d'années (pratiquement 10) pour qu'il en vienne à peu près au niveau de \zsh{}.

Aujourd'hui, c'est le \shell{} par défaut de nombreuses distributions Linux et il commence à devenir très utilisé comme \shell{} de script par défaut. 


\part{la configuration et la ligne de commande}
% !TeX root = optimize-unix.tex

\section{La configuration}
\subsection{Le shell personnel}
La première des configuration est le choix de son shell par défaut sur son compte personnel. C'est très simple :

\lstset{style=shell}
\begin{lstlisting}[caption=changer de shell]
chsh
\end{lstlisting}

Aidons-nous du manuel (sous \netbsd) :

\nicepicture{images/man-chsh.png}{\code{man chsh} sous \netbsd}{ma machine virtuelle}

\subsection{Configurer le prompt}

Sur ma machine virtuelle \netbsd, j'obtiens quelque chose comme ceci:

\nicepicture{images/le-prompt.png}{un prompt sous \netbsd, avec \zsh}{ma machine virtuelle}

Le prompt, ce sont les caractères colorés que l'on voit en début de chaque lignes de commande. Ce prompt m'a aidé, voire sauvé plusieurs fois.  Celui-ci m'affiche le nom de l'utilisateur courant en bleu, de la machine en blanc et du répertoire courant en blanc et gras. Lorsque j'ai des sessions sur plusieurs machines, je vois tout de suite où je me trouve avec son nom. Ensuite, lorsque je me déplace de répertoires en répertoires, je n'ai pas besoin de faire d'éternels \code{pwd} pour savoir où je me trouve. En plus, lorsque je trouve dans un dépôt SVN, j'ai un affichage me donnant les indications sur le répertoire de travail (on ne peut pas le faire sous \cygwin) :

\nicepicture{images/le-prompt-svn.png}{dans un répertoire de travail \osname{SVN}, avec \zsh}{ma machine}

Pour finir, le nom de l'utilisateur change de couleur lorsque je suis en \code{root} :

\nicepicture{images/le-prompt-root.png}{en \code{root} avec \zsh}{ma machine}

Tous les shells interactifs de ma connaissance ont au moins un fichier de configuration exécuté au lancement: avec \zsh, c'est \code{.zhrc}, avec \bash, c'est \code{.bashrc} et avec \csh, c'est \code{.cshrc}. Aussi loin que mes souvenirs remontent, on personnalise le prompt avec la variable \code{PS1} et ce, même pour le MS/DOS.

Voici un hexdump de mon  \code{PS1} :

\begin{lstlisting}[caption=mon prompt]
00000000  25 7b 1b 5b 30 31 3b 33  31 6d 25 7d 25 28 3f 2e  |%{.[01;31m%}%(?.|
00000010  2e 25 3f 25 31 76 20 29  25 7b 1b 5b 33 37 6d 25  |.%?%1v )%{.[37m%|
00000020  7d 25 7b 1b 5b 33 34 6d  25 7d 25 6e 25 7b 1b 5b  |}%{.[34m%}%n%{.[|
00000030  30 30 6d 25 7d 40 25 6d  20 25 34 30 3c 2e 2e 2e  |00m%}@%m %40<...|
00000040  3c 25 42 25 7e 25 62 25  3c 3c 20 24 7b 56 43 53  |<%B%~%b%<< ${VCS|
00000050  5f 49 4e 46 4f 5f 6d 65  73 73 61 67 65 5f 30 5f  |_INFO_message_0_|
00000060  7d 25 23 20 0a                                    |}%# .|
00000065
\end{lstlisting}

% !TeX root = optimize-unix.tex

\section{la ligne de commande}
Pour de multiples raisons déjà plus ou moins évoquées plus haut, j'ai choisi de travailler avec \zsh{} comme shell par défaut. C'est ce que nous allons faire ici, autant sous \freebsd{} que sous \linux{}, tout simplement pour prouver que l'utilisation du shell est assez indépendante du système sous-jacent. Mais comme je sais que certains systèmes viennent avec \bash{} ou \tcsh{}, sans possibilité de modification, je les évoqueraient donc, en particulier \tcsh{} qui est utilisé très souvent, avec \csh{}, pour l'administration. Ce n'est que fortuitement que j'examinerais \ksh{},  autant par manque d'habitude que parce que je ne l'ai jamais rencontré.

\subsection{les boucles}
Il y a \code{while} et \code{for}.

\subsubsection{la boucle \code{for}}
On commence par celle-ci car elle en a dérouté plus d'un. Nous avons, avec \zsh{} et \bash, deux syntaxes essentielles. La première \emph{parcourt} un ensemble de données:

\begin{lstlisting}
for file_name in *.txt
do
	cat $file_name
done
\end{lstlisting}

Il y a la boucle plus classique pour les spécialistes de Java:

\begin{lstlisting}
for ((i=5; i< 8; i++))
do
	echo $i
done
\end{lstlisting}

Avec \tcsh{}, nous aurons:

\begin{lstlisting}
foreach file_name (*.txt)
	cat $file_name
end

foreach i (`seq 5 1 8`)
	echo $i
end
\end{lstlisting}


\subsubsection{la boucle \code{while}}
Elle permet de boucles infinies comme celle-ci avec \zsh{}, \bash{} et \ksh{}:

\begin{lstlisting}
while true; do date ``+%T''; sleep 1; done
\end{lstlisting}

Avec \tcsh{}, nous écrirons en deux lignes:
\begin{lstlisting}
while (1);  date ``+%T''; sleep 1;
end
\end{lstlisting}

\subsection{surprises avec \code{stat}, \code{find} et \code{xargs}}

\subsubsection{\code{stat}}
La commande \code{stat} permet de connaître bon nombre de détails à propos d'un fichier comme ici:

\begin{lstlisting}
bernard@debian7 ~ % stat *
install:
device  70
inode   1837064
mode    16877
nlink   3
uid     1000
gid     1000
rdev    7337007
size    512
atime   1383862259
mtime   1383085660
ctime   1383085660
blksize 16384
blocks  4
link

userstart.tar.gz:
device  70
inode   1837156
mode    33188
nlink   1
uid     1000
gid     1000
rdev    7371584
size    3498854
atime   1383855566
mtime   1383855552
ctime   1383855552
blksize 16384
blocks  6880
link
\end{lstlisting}

On obtient, sous \zsh{}, un résultat totalement identique sous \netbsd{} et sous \linux{}.  Si l'on fait un \code{which stat}, nous obtenons, sur les deux systèmes, le message \code{stat: shell built-in command}. C'est ce qui me plait sous \zsh{}, les commandes non standard comme \code{stat} sont remplacées par des fonctions dont le résultat ne réserve pas de surprise.  Si je veux une sortie plus agréable et n'afficher que la date de dernière modification (\cf \href{http://zsh.sourceforge.net/Doc/Release/Zsh-Modules.html#The-zsh_002fstat-Module}{The zsh/stat module} pour de plus amples explications):

\begin{lstlisting}
bernard@debian7 ~ % stat -F "%Y-%m-%d %T" +mtime -n *
install 2013-10-29 23:27:40
userstart.tar.gz 2013-11-07 21:19:12
bernard@debian7 ~ %
\end{lstlisting}

\begin{lstlisting}
bernard@NBSD-64bits ~ % stat -F "%Y-%m-%d %T" +mtime -n *
install 2013-11-14 09:52:56
userstart.tar.gz 2013-11-08 00:54:06
bernard@NBSD-64bits ~ %
\end{lstlisting}

\subsubsection{réfléchissons un peu}
Grâce à \zsh{}, nous avons une méthode extrêmement portable entre Unix pour afficher des données détaillées des fichiers. Pour l'exemple, prenons le \code{stat} d'origine:

\begin{lstlisting}
bernard@debian7 ~ % /usr/bin/stat --printf="%n %z\n" *
install 2013-10-29 23:27:40.000000000 +0100
userstart.tar.gz 2013-11-07 21:19:12.000000000 +0100
bernard@debian7 ~ %
\end{lstlisting}

\begin{lstlisting}
bernard@NBSD-64bits ~ % /usr/bin/stat -t "%Y-%m-%d %T" -f "%Sc %N" *
2013-11-14 09:52:56 install
2013-11-08 00:54:06 userstart.tar.gz
bernard@NBSD-64bits ~ %
\end{lstlisting}

\subsubsection{\emph{tous} les fichiers du monde}
Si je veux faire la même chose que précédemment, mais sur tous les fichiers de la machine, on peut tomber sur ce message d'erreur:

\begin{lstlisting}
bernard@debian7 ~ % stat -F "%Y-%m-%d %T" +ctime -n $(find / -name "*")
zsh: liste d'arguments trop longue: stat
bernard@debian7 ~ %
\end{lstlisting}

C'est là que \code{xargs} entre en jeu, mais avec un nouveau problème:

\begin{lstlisting}
bernard@debian7 ~ % find / -name "*" | xargs stat -F "%Y-%m-%d %T" +ctime -n
stat: option non valide -- F
Saisissez `` stat --help `` pour plus d'informations.
...
stat: option non valide -- F
Saisissez `` stat --help `` pour plus d'informations.
123 bernard@debian7 ~ %
\end{lstlisting}

La commande \code{xargs} va chercher non pas la fonction de \zsh{} mais le binaire qui se trouve sur le \code{PATH} de la machine. On doit donc faire:

\begin{lstlisting}
bernard@debian7 ~ % find . -name "*" | xargs stat --printf="%n %z\n"
...
./.w3m 2013-11-07 23:07:03.000000000 +0100
./.w3m/configuration 2013-11-07 23:03:48.000000000 +0100
./.w3m/history 2013-11-07 23:06:29.000000000 +0100
./.w3m/cookie 2013-11-07 23:06:29.000000000 +0100
./.viminfo 2013-11-07 23:07:03.000000000 +0100
bernard@debian7 ~ %
\end{lstlisting}

Sous \netbsd:

\begin{lstlisting}
bernard@NBSD-64bits ~ % find . -name "*" | xargs stat -t "%Y-%m-%d %T" -f "%N %Sc"
...
./.lesshst 2013-11-08 01:02:35
./install 2013-11-14 09:52:56
./.zshrc.private~ 2013-11-08 01:13:05
./.viminfo 2013-11-08 01:14:38
./.xinitrc 2013-11-08 01:14:44
./.Xauthority 2013-11-08 01:16:25
bernard@NBSD-64bits ~
\end{lstlisting}

\subsubsection{application pratique}
Sur le serveur, qui est sous \linux, faisons la même chose ou presque, on place la date en premier et c'est la surprise du jour:

\begin{lstlisting}
[bigserver] (688) ~ % find /etc -name  "*" | xargs stat --printf="%z %n\n" | sort
find: "/etc/ssl/private": Permission non accordee
stat: option invalide -- 'o'
Pour en savoir davantage, faites:  stat --help .
[bigserver] (689) ~ %
\end{lstlisting}

En rajoutant l'option \code{-print0} à \code{find}, l'option \code{-0} à \code{xargs}, nous obtenons le bon résultat:

\begin{lstlisting}
[bigserver] (689) ~ % find /etc -name  "*" -print0 | xargs -0  stat --printf="%z %n\n" | sort
...
2013-11-12 10:51:42.041176453 +0100 /etc/php5/conf.d/ldap.ini
2013-11-12 10:55:05.017425662 +0100 /etc/php5/cgi
2013-11-12 10:55:05.017425662 +0100 /etc/php5/cgi/php.ini
2013-11-13 14:55:28.001191244 +0100 /etc/apache2/sites-available/aenercom.preprod.conf
2013-11-13 14:56:35.601352228 +0100 /etc/apache2/sites-available/device.sigrenea.conf
2013-11-13 14:56:35.601352228 +0100 /etc/apache2/sites-enabled
2013-11-13 17:16:04.377217037 +0100 /etc/apache2/sites-available
2013-11-13 17:16:04.377217037 +0100 /etc/phpmyadmin
2013-11-14 01:03:38.589252417 +0100 /etc/php5/conf.d/mysqli.ini
2013-11-14 01:05:14.997350491 +0100 /etc/php5/conf.d
2013-11-14 01:05:14.997350491 +0100 /etc/php5/conf.d/mcrypt.ini
\end{lstlisting}

En fait, les noms de fichier sous \unix{} peuvent contenir beaucoup de caractères étranges en dehors de \code{/}. \code{xargs} prend le caractère \code{LF} comme fin d'enregistrement de la part de son entrée standard. Si jamais un fichier contient ce caractère, plus rien ne va. Les nouvelles options permettent à \code{find} d'utiliser \code{0x00} comme séparateur d'enregistrement et à \code{xargs} de bien l'interpréter.

Il y a aussi une autre explication, depuis bien longtemps les outils \GNU{} fonctionnent comme ceci et ce n'est que très récemment que le couple \code{find}/\code{xargs} fonctionne ainsi.

Après toutes ces considérations, on constate que le 14 Novembre 2013, un peu après 1 heure du matin, quelqu'un a modifié les fichiers \code{/etc/php5/conf.d/mysqli.ini}  et \code{/etc/php5/conf.d/mcrypt.ini}, tout simplement pour remplacer les commentaires de type shell par des commentaires de type fichier ini.

Après une attaque du serveur, il est intéressant de faire le même exercice sur les répertoires vitaux comme \code{/bin}. Pour éviter des listings trop important, on limite la sortie à l'année 2013 et on fait une jolie boucle:

\begin{lstlisting}
[bigserver] (694) ~ % for d in /bin /sbin /lib /lib32 /usr/bin /usr/sbin /usr/lib /usr/lib32; do
find $d -name  "*" -print0 | xargs -0  stat --printf="%z %n\n" | egrep "^2013"
done | sort
\end{lstlisting}

Nous obtenons un listing fort long, correspondant aux mises à jour faites le 8 et le 12 Novembre. Maintenant que nous savons que l'attaque a eu lieu avant le 8 Novembre, on sélectionne plus sévèrement:

\begin{lstlisting}
[bigserver] (695) ~ % for d in /bin /sbin /lib /lib32 /usr/bin /usr/sbin /usr/lib /usr/lib32; do
find $d -name  "*" -print0 | xargs -0  stat --printf="%z %n\n" | egrep "^2013-11-0[1-7]"
done | sort
[bigserver] (696) ~ %
\end{lstlisting}

Cependant, rien ne prouve que nous n'avons pas eu de désordres un peu avant ou un peu pendant. Comme le gros des fichiers est dans \code{/usr/lib}, éliminons le de la liste:

\begin{lstlisting}
[bigserver] (695) ~ % for d in /bin /sbin /lib /lib32 /usr/bin /usr/sbin; do
find $d -name  "*" -print0 | xargs -0  stat --printf="%z %n\n" | egrep "^2013"
done | sort
...
[bigserver] (696) ~ %
\end{lstlisting}

Nous n'avons des modifications qu'entre le 8 et le 12 Novembre.

Plus fort encore, afficher les fichiers modifiés ce jour:

\begin{lstlisting}
find / -name  "*" -print0 | xargs -0  stat --printf="%z %n\n" | egrep "^$(date `+%Y-%m-%d')" | sort
\end{lstlisting}

On est débordé par l'affichage des fichiers système de Linux. Pour palier à cet inconvénient, on demande à \code{find} d'abandonner les répertoire \code{/sys} et \code{/proc}:

\begin{lstlisting}
find / \( -path /proc -o -path /sys \) -prune -o -name  "*" -print0 | xargs -0  stat --printf="%z %n\n" | egrep "^$(date `+%Y-%m-%d')" | sort
\end{lstlisting}

\subsection{les surprises de \code{sudo}}
Reprenons l'exemple précédent en redirigeant la sortie standard vers \code{/dev/null}:

\begin{lstlisting}
find / \( -path /proc -o -path /sys \) -prune -o -name  "*" -print0 | xargs -0  stat --printf="%z %n\n" | egrep "^$(date `+%Y-%m-%d')" > /dev/null
\end{lstlisting}

On aura une sortie comme celle-ci:

\begin{lstlisting}
find: "/var/lib/postgresql/9.1/main": Permission non accordee
find: "/var/lib/sudo": Permission non accordee
find: "/var/cache/ldconfig": Permission non accordee
find: "/var/log/exim4": Permission non accordee
find: "/var/log/apache2": Permission non accordee
...
\end{lstlisting}

Pour éliminer les \code{find: ... Permission non accordee}, on utilise \code{sudo}:

\begin{lstlisting}
sudo find / \( -path /proc -o -path /sys \) -prune -o -name  "*" -print0 | xargs -0  stat --printf="%z %n\n" | egrep "^$(date `+%Y-%m-%d')" > /dev/null
\end{lstlisting}

C'est pire:

\begin{lstlisting}
...
stat: impossible d'evaluer ' /root/.aptitude ': Permission non accordee
stat: impossible d'evaluer ' /root/.aptitude/cache ': Permission non accordee
stat: impossible d'evaluer ' /root/.aptitude/config ': Permission non accordee
stat: impossible d'evaluer ' /root/.viminfo ': Permission non accordee
stat: impossible d'evaluer ' /root/.bash_history ': Permission non accordee
...
\end{lstlisting}

Nous avons demandé à \code{sudo} de traité \code{find} et avec le \emph{pipe}, nous demandons à \code{xargs} de traiter les lignes de sorties avec \code{stat}. Ce dernier récupère un nom de fichier et le traite comme tel mais comme il n'est pas lancé avec \code{sudo}, nous avons ces erreurs. Essayons ceci:

\begin{lstlisting}
sudo find / \( -path /proc -o -path /sys \) -prune -o -name  "*" -print0 | xargs -0  sudo stat --printf="%z %n\n" | egrep "^$(date `+%Y-%m-%d')" > /dev/null
\end{lstlisting}

C'est pas mieux, autant sous \linux{} que sous \netbsd{}:

\begin{lstlisting}
sudo: unable to execute /usr/bin/stat: Argument list too long
sudo: unable to execute /usr/bin/stat: Argument list too long
sudo: unable to execute /usr/bin/stat: Argument list too long
sudo: unable to execute /usr/bin/stat: Argument list too long
sudo: unable to execute /usr/bin/stat: Argument list too long
sudo: unable to execute /usr/bin/stat: Argument list too long
\end{lstlisting}

Il faut bien l'avouer, je ne sais pas quoi dire de plus ici - sinon noter un \emph{TODO: comprendre ce qui ce passe}. Ce qui est bien avec \unix{}, c'est qu'il y a toujours un moyen de s'en sortir. On remarquera quelques différences entre les mondes \linux{} et \BSD{}, en particulier avec \code{man sh}, où le premier nous renvoie sur \bash{} alors que le second traite bien directement de \sh{}. Dans tous les cas, \code{sudo sh -c ``...''} est notre amie et nous obtenons avec  \netbsd{}:

\begin{lstlisting}
bernard@NBSD-64bits ~ % sudo sh -c "find / \( -path /proc -o -path /sys \) -prune -o -name '*' -type f | xargs stat -t '%Y-%m-%d %T' -f '%Sc %N' | egrep '^$(date "+%Y-%m-%d")' | sort"
2013-11-15 08:55:19 /var/run/dmesg.boot
2013-11-15 08:55:22 /var/log/messages
2013-11-15 08:55:22 /var/run/ntpd.pid
2013-11-15 08:55:22 /var/run/powerd.pid
2013-11-15 08:55:22 /var/run/sshd.pid
\end{lstlisting}

Et sous \linux{}:

\begin{lstlisting}
bernard@debian7 ~ % sudo sh -c "find / \( -path /proc -o -path /sys \) -prune -o -name  '*' -print0 | xargs -0  stat --printf='%z %n\n' | egrep '^$(date "+%Y-%m-%d")'"
2013-11-15 09:54:42.000000000 +0100 /var/lib/misc/statd.status
2013-11-15 09:54:41.000000000 +0100 /var/lib/urandom/random-seed
2013-11-15 09:55:09.000000000 +0100 /var/lib/dhcp/dhclient.em0.leases
2013-11-15 09:54:49.000000000 +0100 /var/lib/exim4
2013-11-15 09:54:49.000000000 +0100 /var/lib/exim4/config.autogenerated
2013-11-15 09:54:45.000000000 +0100 /var/lib/postgresql/9.1/main
2013-11-15 09:54:46.000000000 +0100 /var/lib/postgresql/9.1/main/global
...
\end{lstlisting}

\subsection{\POSIX{} et \GNU{}}
Profitons d'un moment de calme pour remarquer que de nombreuses commandes se comportent de manière très standard entre différents systèmes, y compris parfois, sous \msdos{}. Tout cela vient de \POSIX{} ou de \GNU{}.

Les guerres de religions qui opposent parfois violemment les mondes \BSD{} et \linux{}, les supporters de \code{Vi} et \code{Emacs}, \ldots finissent par être absorbées avec le temps et seuls quelques irréductibles les raniment, souvent plus pour s'exposer aux yeux (blasés maintenant) du petit monde concerné. Seule reste l'opposition farouche entre tenants du libre et leurs opposants.

Ici, nous avons utilisé \code{find} de la même manière sous \linux{} et sous \netbsd{} ce qui n'a pas été toujours le cas, de même, \sh{} se comporte de manière identique à quelques octets près sur les deux systèmes, ce qui n'était pas forcément vrai il y a quelques années. Pour revenir à \code{find}, nous avions un paquet de compatibilité \GNU{} disponible sur plusieurs \BSD{} qui reprenait le \code{find} que nous connaissons maintenant et l'on pouvait différencier \code{gfind} de \code{bsdfind}\footnote{A vérifier dans les détails.}.

\part{les scripts et les exemples}
% !TeX root = optimize-unix.tex

\section{Les scripts \shell}
La magie des \shells est infinie, ils nous permettent en effet de créer des programmes complets, complexes... parfois aux limites du lisible. On les appelle \emph{scripts} pour les opposer aux applications généralement créées à partir de langages compilés mais cela ne devrait rien changer au fait qu'ils doivent être conçus avec un soin égal à celui apporté aux autres langages comme \emph{C/C++}, \emph{Java}...

Dans tout ce qui suit, il ne faut pas perdre de vue que le \shell est une \emph{coquille} entourant le noyau d'\unix. Certains aspects des \shells ne font que recouvrir des appels systèmes.

\subsection{structure des scripts}
Ce qui est décrit ici est valable autant pour des langages interprétés comme l'horrible \emph{Perl}\footnote{je ne suis pas objectif, mais quand même...}, le sublime \emph{Scheme}\footnote{là, je me sens plus objectif... ou presque.}, le célèbre \emph{Python} que pour n'importe quel \emph{shell}.

La première ligne d'un script est le \emph{shebang}. Cette ligne est très importante car elle indique de manière sûre quel interpréteur il doit utiliser pour exécuter le corps du script. Voici quelques exemples :

\begin{description}
\item[sh] : \code{\#!/bin/sh}
\item[bash] : \code{\#!/bin/sh}
\item[Perl] : \code{\#!/usr/bin/env perl}
\item[Python 2.7] : \code{\#!/usr/bin/env python2.7}
\item[Python] : \code{\#!/usr/bin/env python}
\item[awk] : \code{\#!/bin/awk -f}
\end{description}

Les deux caractères \code{\#!} sont considérés comme un nombre magique par le système d'exploitation qui comprend immédiatement qu'il doit utiliser le script dont le nom et les arguments suivent les deux caractères.

Dans un \shell interactif, l'exécution d'un script suit l'algorithme suivant :

\begin{lstlisting}
fork ();
if (child) {
	open(script);
	switch(magic_number) {
		case 0x7f'ELF':
			exec_binaire();
			break;
		case `\#!':
			load_shell(first_line);
			exec_binaire(shellname, args);
			break;
		...
	}
} else {
	wait(child);
}		
\end{lstlisting}

\subsection{choisir son \shell}
Par tradition autant que par prudence, on écrit ses scripts \shell pour le \shell d'origine, soit \sh. Par prudence car on est certain qu'il sera présent sur la machine même si elle démarre en mode dégradé. Cependant, beaucoup de scripts sont \emph{applicatifs} et ne pourront pas fonctionner en mode dégradé. Autant se servir d'un \shell plus complet comme \bash. 

\subsection{les paramètres des scripts}
Les paramètres, leur nombre et leur taille n'ont de limites que de l'ordre de la dizaine de Ko. Il faut donc pouvoir y accéder. Le paramètre \code{\$0} est le nom du script parfois avec le nom du répertoire. Les neufs suivants sont nommés \code{\$1}, ..., \code{\$9}. Pour accéder aux autres il faut ruser un peu avec l'instruction \code{shift}.

\subsection{tests et boucles}
Les tests se font avec \code{if} de cette manière :
\begin{lstlisting}
if condition
then
	...
else
	...
fi	
\end{lstlisting}

Dans le même ordre d'idée, nous avons le \code{while} :
\begin{lstlisting}
while condition
do
done
\end{lstlisting}

La construction des \code{condition} est tout un art, d'autant plus qu'en lieu et place du \code{if} nous pouvons écrire :
\begin{lstlisting}
condition && condition_true && ...
\end{lstlisting}

ou bien :
\begin{lstlisting}
condition || condition_false 
\end{lstlisting}

Nous avons aussi une boucle \code{for} :
\begin{lstlisting}
for index in ensemble
do
	...
done
\end{lstlisting}

La détermination de \code{ensemble} est assez naturelle comme par exemple avec \code{\$(ls *.java)}. Mais il faut être prudent: selon les \shells les résultats peuvent différer.

\subsection{conditions, valeurs de retour des programmes}

Tout les programmes sous \unix s'achèvent par un \code{return EXIT\_CODE} ou bien un \code{exit(EXIT\_CODE)} bien senti. La valeur \code{EXIT\_CODE} est renvoyée au programme appelant, notre \shell. On peut le récupérer depuis la variable \code{\$\#?} puis étudier le cas :
\begin{lstlisting}
myprogram arg1 arg2 ...
case \$\#? in
	0)
		its-okayyy
		;;
	1)
		bad_parameters
		;;
	2|3|4)
		cant-open-files
		;;
	*)
		unknow-error
		;;
esac
\end{lstlisting}

\unix considère que la valeur de retour \code{0} est signe que tout va bien et que tout autre valeur exprime une condition d'erreur. On peut donc utiliser cette propriété ainsi :

\begin{lstlisting}
myprogram arg1 arg2 ... || onerror ``Error code \$\#?''
\end{lstlisting}

\subsection{redirections}

\subsection{tubes ou \emph{pipes}}

% !TeX root = optimize-unix.tex

\section{Exemples de manipulation de texte}

\subsection{Des stats}
Voici un extrait d'un fichier \code{/var/log/messages}:

\begin{lstlisting}
Nov  3 10:16:19 localhost org.gnome.zeitgeist.SimpleIndexer[2637]: ** \ldots
Nov  3 10:16:34 localhost org.freedesktop.FileManager1[2637]: Initializing \ldots
Nov  3 10:16:34 localhost nautilus: [N-A] Nautilus-Actions Menu Extender 3.2\ldots
Nov  3 10:16:34 localhost org.freedesktop.FileManager1[2637]: Initializing naut\ldots
Nov  3 10:16:34 localhost nautilus: [N-A] Nautilus-Actions Tracker 3.2.3 initializing\ldots
\end{lstlisting}

Nous voulons déterminer les moments les plus actifs de ce fichier avec une granularité de une heure. La manipulation est simple:

\begin{description}
    \item[afficher le fichier] \code{cat file-name},
    \item[découper le ficher] \code{cut -d ':' -f 1},
    \item[trier le fichier] \code{sort},
    \item[compter les occurences] \code{uniq -c},
    \item[trier en décroissant] \code{sort -n}.
\end{description}

Ce qui nous donne la commande:

\begin{lstlisting}
cat $file-name |
    cut -d ':' -f 1 |
        sort |
            uniq -c |
                sort -n
\end{lstlisting}

On obtient rapidement un script (\code{stat1.sh}) à partir de cette ligne de com\-man\-de :

\lstinputlisting{code/stat1.sh}

On peut tester :

\begin{lstlisting}
$ ./stat1.sh messages-1 /var/log/messages /var/log/messages.1
...
  152 Nov  4 10
  155 Oct 27 11
  156 Oct 28 09
  164 Oct 28 17
  186 Oct 25 14
  213 Oct 28 11
  216 Nov  3 10
  260 Oct 28 10
  636 Oct 26 15
  774 Oct 28 07
 1770 Nov  3 09
 3844 Oct 28 14
26201 Oct 28 15
\end{lstlisting}

\subsection{Peut-on faire mieux?}
Bien sûr! On peut avoir d'autres options que la simple aide, on peut aussi gérer correctement les erreurs, les \emph{signaux \unix}\ldots

\subsubsection{Les options}
Depuis longtemps il existe une norme \POSIX{} permettant de gérer les options de la ligne
de commande. Malheureusement, il fut une époque où la norme avait beaucoup de variantes
ce qui m'a poussé à faire ma propre gestion de ces paramètres.

Voici ma méthode, facile à mémoriser mais pas parfaite et un peu lourde:

\begin{description}
    \item[l'aide] créer une fonction \code{dohelp} comme dans l'exemple précédent;
    le nom \code{do\-help} per\-met d'éviter un clash avec une éventuelle commande \code{help}.
    \item[s'assurer de l'existence de paramètres] il suffit de faire le test \code{[ \$\# -eq 0 ]} et
    exécuter le code nécessaire.
    \item[vider la liste des paramètres] une boucle \code{while [ \$\# -ne 0 ]} fait l'affaire.
\end{description}

Voici un exemple plus parlant (script \code{stat2.sh}):
\lstinputlisting{code/stat2.sh}

Et maintenant avec le \code{getopts}\footnote{soyez prudents avec les (très) anciennes versions de \emph{Red Hat}}:
\lstinputlisting{code/stat3.sh}

En fait \code{getopts} ne sait traiter que les \emph{options courtes} et classiques d'\unix. Les \emph{options longues}
à la mode \linux{} ne sont pas supportées. L'avantage de \code{getopts} est son mode de fonctionnement
assez simple. Son inconvénient principal est d'être très spécifique à \bash{} même si \POSIX{} le soutient, ce qui fait qu'il
n'est pas forcément disponible partout.

Pour avoir les \emph{options longues}, il faut utiliser l'outil \GNU{} \code{getopt} (sans le \code{s} de fin)\footnote{voir cette discussion sur \href{http://stackoverflow.com/questions/402377/using-getopts-in-bash-shell-script-to-get-long-and-short-command-line-options}{StackOverflow}}.
Je reste donc sur ma méthode qui n'est finalement ni meilleure ni pire.

\subsubsection{Avec \code{bash}}
On peut profiter des avantages de \bash (boucles, tableaux, \ldots) com\-me dans ce script (qui ne fonctionne \textbf{pas} avec \sh) :

\lstinputlisting{code/stat4.sh}

Conrairement aux précédents, si une tranche horaire n'est pas représentée dans les fichiers logs passés en paramètres, elle sera tout de même affichée avec la valeur 0.

\part{commandes utiles}
% !TeX root = optimize-unix.tex

\section{les noms de fichiers}

Les commandes les plus utiles sont \code{dirname} et \code{basename}. La première renvoie le repertoire du nom fichier et la seconde renvoie simplement le nom de base comme ici:

\begin{lstlisting}
$ which firefox 
/usr/local/bin/firefox
$ basename $(which firefox)
firefox
$ dirname $(which firefox) 
/usr/local/bin
$
\end{lstlisting}

On en profite pour présenter l'indispensable \code{which} qui donne le nom complet d'une application se trouvant dans le \code{PATH}.

\section{cherche et remplace}
Les outils de base sont \code{egrep}, \code{sed} et \code{tr}\footnote{L'outil \code{awk}, est, à mon humble avis, à reléguer dans les musées.}. Ces trois outils utilisent les expressions régulières. Ces expressions régulières ressemblent fortement à ce que l'on trouve dans Perl, Python, Java et les autres. La différence fondamentale à ne pas oublier: les expressions régulières des outils \GNU{} sont rapides, efficaces, les autres\ldots beaucoup moins\footnote{L'introduction du \emph{backtracking} peut détruire complètement les performances}.

Les expressions régulières méritent une formation complète car elles ne sont pas vraiment intuitives. De plus, les variations qui existent entre \POSIX, \GNU, \BSD, les \shells{} qui en rajoutent parfois, sans compter que certains outils, certaines distributions n'ont pas leurs outils vraiment à jour (\cf{} \RedHat{}).

\subsection{cherche}
Pour la recherche, nous avons \code{grep} et \code{egrep}. En fait, la plupart du temps, \code{egrep} équivaut à \code{grep -E} permettant l'utilisation des expressions régulières dites étendues ou \POSIX{} dont la documentation se trouve dans \code{man 7 regex} sous \Debian.



\printindex
\end{document}
