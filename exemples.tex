% !TeX root = optimize-unix.tex

\section{Exemples de manipulation de texte}

\subsection{Des stats}
Voici un extrait d'un fichier \code{/var/log/messages}:

\begin{lstlisting}
Nov  3 10:16:19 localhost org.gnome.zeitgeist.SimpleIndexer[2637]: ** \ldots
Nov  3 10:16:34 localhost org.freedesktop.FileManager1[2637]: Initializing \ldots
Nov  3 10:16:34 localhost nautilus: [N-A] Nautilus-Actions Menu Extender 3.2\ldots
Nov  3 10:16:34 localhost org.freedesktop.FileManager1[2637]: Initializing naut\ldots
Nov  3 10:16:34 localhost nautilus: [N-A] Nautilus-Actions Tracker 3.2.3 initializing\ldots
\end{lstlisting}

Nous voulons déterminer les moments les plus actifs de ce fichier avec une granularité de une heure.
La manipulation est simple:

\begin{description}
    \item[afficher le fichier] \code{cat file-name},
    \item[découper le ficher] \code{cut -d ':' -f 1},
    \item[trier le fichier] \code{sort},
    \item[compter les occurences] \code{uniq -c},
    \item[trier en décroissant] \code{sort -n}.
\end{description}

Ce qui nous donne la commande:

\begin{lstlisting}
cat $file-name |
    cut -d ':' -f 1 |
        sort |
            uniq -c |
                sort -n
\end{lstlisting}

On obtient rapidement un script (\code{stat1.sh}) à partir de cette ligne de com\-man\-de :

\lstinputlisting{code/stat1.sh}

On peut tester :

\begin{lstlisting}
$ ./stat1.sh messages-1 /var/log/messages /var/log/messages.1
...
  152 Nov  4 10
  155 Oct 27 11
  156 Oct 28 09
  164 Oct 28 17
  186 Oct 25 14
  213 Oct 28 11
  216 Nov  3 10
  260 Oct 28 10
  636 Oct 26 15
  774 Oct 28 07
 1770 Nov  3 09
 3844 Oct 28 14
26201 Oct 28 15
\end{lstlisting}

\subsection{Peut-on faire mieux?}
Bien sûr! On peut avoir d'autres options que la simple aide, on peut aussi gérer correctement les erreurs, les \emph{signaux \unix}\ldots

\subsubsection{Les options}
Depuis longtemps il existe une norme \POSIX{} permettant de gérer les options de la ligne
de commande. Malheureusement, il fut une époque où la norme avait beaucoup de variantes
ce qui m'a poussé à faire ma propre gestion de ces paramètres.

Voici ma méthode, facile à mémoriser mais pas parfaite et un peu lourde:

\begin{description}
    \item[l'aide] créer une fonction \code{dohelp} comme dans l'exemple précédent;
    le nom \code{do\-help} per\-met d'éviter un clash avec une éventuelle commande \code{help}.
    \item[s'assurer de l'existence de paramètres] il suffit de faire le test \code{[ \$\# -eq 0 ]} et
    exécuter le code nécessaire.
    \item[vider la liste des paramètres] une boucle \code{while [ \$\# -ne 0 ]} fait l'affaire.
\end{description}

Voici un exemple plus parlant (script \code{stat2.sh}):
\lstinputlisting{code/stat2.sh}

Et maintenant avec le \code{getopts}\footnote{soyez prudents avec les (très) anciennes versions de \emph{Red Hat}}:
\lstinputlisting{code/stat3.sh}

En fait \code{getopts} ne sait traiter que les \emph{options courtes} et classiques d'\unix. Les \emph{options longues}
à la mode \linux{} ne sont pas supportées. L'avantage de \code{getopts} est son mode de fonctionnement
assez simple. Son inconvénient principal est d'être très spécifique à \bash{} même si \POSIX{} le soutient, ce qui fait qu'il
n'est pas forcément disponible partout.

Pour avoir les \emph{options longues}, il faut utiliser l'outil \GNU{} \code{getopt} (sans le \code{s} de fin)\footnote{voir cette discussion sur \href{http://stackoverflow.com/questions/402377/using-getopts-in-bash-shell-script-to-get-long-and-short-command-line-options}{StackOverflow}}. 
Je reste donc sur ma méthode qui n'est finalement ni meilleure ni pire.

\subsubsection{Avec \code{bash}}
On peut profiter des avantages de \bash (boucles, tableaux, \ldots) com\-me dans ce script (qui ne fonctionne \textbf{pas} avec \sh) :

\lstinputlisting{code/stat4.sh}

Conrairement aux précédents, si une tranche horaire n'est pas représentée dans les fichiers logs passés en paramètres, elle sera tout de même affichée avec la valeur 0. 
