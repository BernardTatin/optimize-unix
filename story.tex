% !TeX root = optimize-unix.tex

\section{Une histoire d'\unix}
Voici une (rapide) histoire d'\unix{}, choisie parmi d'autres, parmi celles qui évoluent avec le temps autant parce que des personnages hauts en couleur et ayant réussi à voler la vedette à de plus modestes collègues se font effacer eux-même par de plus brillants qu'eux, soit parce que, vieillissant ils se laissent aller à des confidences inattendues.

En nous basant sur \href{http://www.tuteurs.ens.fr/unix/histoire.html}{Brève histoire d'\unix}, on rappelle que \emph{AT\&T} travaillait à la fin des années 60, sur un système d'exploitation \href{http://fr.wikipedia.org/wiki/Multics}{\multics} qui devait révolutionner l'histoire de l'informatique. Si révolution il y eut, ce fut dans les esprits: de nombreux concepts de ce système ont influencés ses successeurs, dont \unix. Ken Thompson et Dennis Ritchie des fameux \emph{Bell Labs} et qui travaillaient (sans grande conviction, semble-t-il) sur \multics, décidèrent de lancer leur propre projet d'OS : 

\begin{Quote}
baptisé initialement UNICS (UNiplexed Information and Computing Service) jeu de mot avec "eunuchs' (eunuque) pour "un \multics emasculé", par clin d'œil au projet \multics, qu'ils jugeaient beaucoup trop compliqué. Le nom fut ensuite modifié en \unix\footnote{\cf{} l'article \href{http://fr.wikipedia.org/wiki/Multics}{\multics} de Wikipedia}.
\end{Quote}

\begin{Quote}
L'essor d'\unix est très fortement lié à un langage de programmation, le C. À l'origine, le premier \unix était écrit en assembleur, puis Ken Thompson crée un nouveau langage, le B. En 1971, Dennis Ritchie écrit à son tour un nouveau langage, fondé sur le B, le C. Dès 1973, presque tout \unix est réécrit en C. Ceci fait probablement d'\unix le premier système au monde écrit dans un langage portable, c'est-à-dire autre chose que de l'assembleur\footnote{\cf{}  \href{http://www.tuteurs.ens.fr/unix/histoire.html}{Brève histoire d'\unix}}.
\end{Quote}

Ce que j'ai surtout retenu de tout cela,  c'est qu'\unix a banalisé autant l'utilisation des stations de travail connectées en réseau que le concept de \emph{shell}, des système de fichiers hiérarchisés, des périphériques considérés comme de simples fichiers, concepts repris (et certainement améliorés) à \multics comme à d'autres. Pour moi, la plus grande invention d'\unix, c'est le langage C qui permet l'écriture des systèmes d'exploitations et des logiciels d'une manière très portable. N'oublions pas qu'aujourd'hui encore, C (mais pas C++) est un des langages les plus portable, même s'il commence à être concurrencé par Java par exemple.
