% !TeX root = optimize-unix.tex

\section{les noms de fichiers}

Les commandes les plus utiles sont \code{dirname} et \code{basename}. La première renvoie le repertoire du nom fichier et la seconde renvoie simplement le nom de base comme ici:

\begin{lstlisting}
$ which firefox 
/usr/local/bin/firefox
$ basename $(which firefox)
firefox
$ dirname $(which firefox) 
/usr/local/bin
$
\end{lstlisting}

On en profite pour présenter l'indispensable \code{which} qui donne le nom complet d'une application se trouvant dans le \code{PATH}.

\section{cherche et remplace}
Les outils de base sont \code{egrep}, \code{sed} et \code{tr}\footnote{L'outil \code{awk}, est, à mon humble avis, à reléguer dans les musées.}. Ces trois outils utilisent les expressions régulières. Ces expressions régulières ressemblent fortement à ce que l'on trouve dans Perl, Python, Java et les autres. La différence fondamentale à ne pas oublier: les expressions régulières des outils \GNU{} sont rapides, efficaces, les autres\ldots beaucoup moins\footnote{L'introduction du \emph{backtracking} peut détruire complètement les performances}.



Les expressions régulières méritent une formation complète car elles ne sont pas vraiment intuitives. De plus, les variations qui existent entre \POSIX, \GNU, \BSD, les \shells{} qui en rajoutent parfois, sans compter que certains outils, certaines distributions n'ont pas leurs outils vraiment à jour (\cf{} \emph{Red Hat}).
