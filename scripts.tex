% !TeX root = optimize-unix.tex

\section{Les scripts \shell}
La magie des \shells est infinie, ils nous permettent en effet de créer des programmes complets, complexes... parfois aux limites du lisible. On les appelle \emph{scripts} pour les opposer aux applications généralement créées à partir de langages compilés mais cela ne devrait rien changer au fait qu'ils doivent être conçus avec un soin égal à celui apporté aux autres langages comme \emph{C/C++}, \emph{Java}...

Dans tout ce qui suit, il ne faut pas perdre de vue que le \shell est une \emph{coquille} entourant le noyau d'\unix. Certains aspects des \shells ne font que recouvrir des appels systèmes.

\subsection{Structure des scripts}
Ce qui est décrit ici est valable autant pour des langages interprétés comme l'horrible \emph{Perl}\footnote{je ne suis pas objectif, mais quand même...}, le sublime \emph{Scheme}\footnote{là, je me sens plus objectif... ou presque.}, le célèbre \emph{Python} que pour n'importe quel \emph{shell}.

La première ligne d'un script est le \emph{shebang}. Cette ligne est très importante car elle indique de manière sûre quel interpréteur il doit utiliser pour exécuter le corps du script. Voici quelques exemples :

\begin{description}
\item[sh] : \code{\#!/bin/sh}
\item[bash] : \code{\#!/bin/sh}
\item[Perl] : \code{\#!/usr/bin/env perl}
\item[Python 2.7] : \code{\#!/usr/bin/env python2.7}
\item[Python] : \code{\#!/usr/bin/env python}
\item[awk] : \code{\#!/bin/awk -f}
\end{description}

Les deux caractères \code{\#!} sont considérés comme un nombre magique par le système d'exploitation qui comprend immédiatement qu'il doit utiliser le script dont le nom et les arguments suivent les deux caractères.

Dans un \shell interactif, l'exécution d'un script suit l'algorithme suivant :

\begin{lstlisting}
fork ();
if (child) {
	open(script);
	switch(magic_number) {
		case 0x7f'ELF':
			exec_binaire();
			break;
		case `\#!':
			load_shell(first_line);
			exec_binaire(shellname, args);
			break;
		...
	}
} else {
	wait(child);
}
\end{lstlisting}

\subsection{Choisir son \shell}
Par tradition autant que par prudence, on écrit ses scripts \shell pour le \shell d'origine, soit \sh. Par prudence car on est certain qu'il sera présent sur la machine même si elle démarre en mode dégradé. Cependant, beaucoup de scripts sont \emph{applicatifs} et ne pourront pas fonctionner en mode dégradé. Autant se servir d'un \shell plus complet comme \bash.

\subsection{Les paramètres des scripts}

Les paramètres, leur nombre et leur taille n'ont de limites que de l'ordre de la dizaine de Ko. Il faut donc pouvoir y accéder. Le paramètre \code{\$0} est le nom du script parfois avec le nom du répertoire. Les neufs suivants sont nommés \code{\$1}, ..., \code{\$9}. Pour accéder aux autres il faut ruser un peu avec l'instruction \code{shift}.

\subsection{Tests et boucles}
Les tests se font avec \code{if} de cette manière :
\begin{lstlisting}
if condition
then
	...
else
	...
fi
\end{lstlisting}

Dans le même ordre d'idée, nous avons le \code{while} :
\begin{lstlisting}
while condition
do
done
\end{lstlisting}

La construction des \code{condition} est tout un art, d'autant plus qu'en lieu et place du \code{if} nous pouvons écrire :
\begin{lstlisting}
condition && condition_true && ...
\end{lstlisting}

ou bien :
\begin{lstlisting}
condition || condition_false
\end{lstlisting}

Nous avons aussi une boucle \code{for} :
\begin{lstlisting}
for index in ensemble
do
	...
done
\end{lstlisting}

La détermination de \code{ensemble} est assez naturelle comme par exemple avec \code{\$(ls *.java)}. Mais il faut être prudent: selon les \shells les résultats peuvent différer.

\subsection{Conditions, valeurs de retour des programmes}

Tout les programmes sous \unix s'achèvent par un \code{return EXIT\_CODE} ou bien un \code{exit(EXIT\_CODE)} bien senti. La valeur \code{EXIT\_CODE} est renvoyée au programme appelant, notre \shell. On peut le récupérer depuis la variable \code{\$\#?} puis étudier le cas :
\begin{lstlisting}
myprogram arg1 arg2 ...
case \$\#? in
	0)
		its-okayyy
		;;
	1)
		bad_parameterzzz
		;;
	2|3|4)
		cant-open-filezzz
		;;
	*)
		unknow-error
		;;
esac
\end{lstlisting}

\unix considère que la valeur de retour \code{0} est signe que tout va bien et que tout autre valeur exprime une condition d'erreur. On peut donc utiliser cette propriété ainsi :

\begin{lstlisting}
myprogram arg1 arg2 ... || onerror ``Error code \$\#?''
\end{lstlisting}

\subsection{Redirections et tubes (ou \emph{pipes})}
Le premier piège dans l'utilisation des tubes dans un scripts est simple: pour chaque tube, on crée un nouveau processus. Ainsi le script suivant ne renvoie pas le résultat escompté:

\lstinputlisting{code/bad1.sh}

\begin{lstlisting}
$ ./bad1.sh
Il y a 0 fichiers dans /bin
\end{lstlisting}

La variable compteur fait partie de l'environnement de \code{bad1.sh}. Lorsque la boucle \code{while} se lance, elle est dans un nouveau contexte et sa modification se perd à la fin de la boucle. On corrige de cette manière :

\lstinputlisting{code/not-so-bad1.sh}

\begin{lstlisting}
 ./not-so-bad1.sh
Il y a 173 fichiers dans /bin
\end{lstlisting}
