% !TeX root = optimize-unix.tex

\section{Les scripts \shell}
La magie des \shells est infinie, ils nous permettent en effet de créer des programmes complets, complexes... parfois aux limites du lisible. On les appelle \emph{scripts} pour les opposer aux applications généralement créées à partir de langages compilés mais cela ne devrait rien changer au fait qu'ils doivent être conçus avec un soin égal à celui apporté aux autres langages comme \emph{C/C++}, \emph{Java}...

Dans tout ce qui suit, il ne faut pas perdre de vue que le \shell est une \emph{coquille} entourant le noyau d'Unix. Certains aspects des \shells ne font que recouvrir des appels systèmes.

\subsection{structure des scripts}
Ce qui est décrit ici est valable autant pour des langages interprétés comme l'horrible \emph{Perl}\footnote{je ne suis pas objectif, mais quand même...}, le sublime \emph{Scheme}\footnote{là, je me sens plus objectif... ou presque.}, le célèbre \emph{Python} que pour n'importe quel \emph{shell}.

La première ligne d'un script est le \emph{shebang}. Cette ligne est très importante car elle indique de manière sûre quel interpréteur il doit utiliser pour exécuter le corps du script. Voici quelques exemples :

\begin{description}
\item[sh] : \code{\#!/bin/sh}
\item[bash] : \code{\#!/bin/sh}
\item[Perl] : \code{\#!/usr/bin/env perl}
\item[Python 2.7] : \code{\#!/usr/bin/env python2.7}
\item[Python] : \code{\#!/usr/bin/env python}
\item[awk] : \code{\#!/bin/awk -f}
\end{description}

Les deux caractères \code{\#!} sont considérés comme un nombre magique par le système d'exploitation qui comprend immédiatement qu'il doit utiliser le script dont le nom et les arguments suivent les deux caractères.

Dans un \shell interactif, l'exécution d'un script suit l'algorithme suivant :

\begin{lstlisting}
fork ();
if (child) {
	open(script);
	switch(magic_number) {
		case 0x7f'ELF':
			exec_binaire();
			break;
		case `\#!':
			load_shell(first_line);
			exec_binaire(shellname, args);
			break;
		...
	}
} else {
	wait(child);
}		
\end{lstlisting}


